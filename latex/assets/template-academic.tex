\documentclass[12pt, a4paper]{article}

% PAQUETES NECESARIOS
\usepackage[utf8]{inputenc}
\usepackage{amsmath}
\usepackage{amssymb}
\usepackage{graphicx}
\usepackage{booktabs}
\usepackage{hyperref}
\usepackage[margin=2.5cm]{geometry}
\usepackage{listings}
\usepackage{xcolor}

% CONFIGURACIÓN DE LISTINGS PARA PYTHON
\definecolor{codegreen}{rgb}{0,0.6,0}
\definecolor{codegray}{rgb}{0.5,0.5,0.5}
\definecolor{codepurple}{rgb}{0.58,0,0.82}
\definecolor{backcolour}{rgb}{0.95,0.95,0.92}

\lstdefinestyle{mystyle}{
    backgroundcolor=\color{backcolour},
    commentstyle=\color{codegreen},
    keywordstyle=\color{magenta},
    numberstyle=\tiny\color{codegray},
    stringstyle=\color{codepurple},
    basicstyle=\footnotesize\ttfamily,
    breakatwhitespace=false,
    breaklines=true,
    captionpos=b,
    keepspaces=true,
    numbers=left,
    numbersep=5pt,
    showspaces=false,
    showstringspaces=false,
    showtabs=false,
    tabsize=2
}
\lstset{style=mystyle, language=Python}

% PORTADA
\title{Informe Académico: [Título]}
\author{[Autor]}
\date{\today}

\begin{document}
\sloppy

\maketitle

\begin{abstract}
[Resumen del documento]
\end{abstract}

\tableofcontents
\newpage

\section{Introducción}

[Contenido de la introducción]

\section{Sección Principal}

[Contenido principal]

\begin{lstlisting}[caption=Código Python]
def hello_world():
    print("Hello World")
\end{lstlisting}

\begin{table}[h!]
\centering
\caption{Ejemplo de tabla}
\begin{tabular}{@{}lcc@{}}
\toprule
\textbf{Columna 1} & \textbf{Columna 2} & \textbf{Columna 3} \\ \midrule
Dato 1 & Dato 2 & Dato 3 \\
Dato 4 & Dato 5 & Dato 6 \\ \bottomrule
\end{tabular}
\end{table}

\section{Conclusión}

[Conclusión]

\begin{thebibliography}{9}
\bibitem{ejemplo}
Ejemplo de referencia.
\end{thebibliography}

\end{document}